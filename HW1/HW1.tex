\documentclass{article}
\usepackage[utf8]{inputenc}


%%Let's you change margins
\usepackage[left=1in,right=1in,top=1in,bottom=1in]{geometry}
\usepackage{graphicx}
\usepackage{tikz}
\usepackage{tikz-cd}
%%Math symbols, proof environments
\usepackage{amsmath,amsthm,amssymb}
\newcommand*{\twoheadrightarrowtail}{\mathrel{\rightarrowtail\kern-1.9ex\twoheadrightarrow}}
%%Make links, "colorlinks" makes it so the links show up in a different color from the text
\usepackage[colorlinks]{hyperref}

%%Use this package for matrices
\usepackage{array}

%%Defining commands for common sets
\newcommand{\R}{\mathbb{R}} %Real numbers
\DeclareMathOperator{\ord}{ord}
\DeclareMathOperator{\lcm}{lcm}
\DeclareMathOperator{\Orb}{Orb}

\newcommand{\N}{\mathbb{N}}
\newcommand{\Q}{\mathbb{Q}} %Rational numbers
\newcommand{\Z}{\mathbb{Z}} %Integers
\newtheorem{problem}{Problem}[section]
\newtheorem{conjecture}{Conjecture}[section]
\newtheorem{lemma}{Lemma}[section]
\newtheorem{definition}{Definition}[section]

\title{Math 197 Week 3 Exercises} 

\author{George Luan}

\date{January 23, 2024} %You can also manually type in a date

\begin{document}

\maketitle

\section*{Problem 2.2}
\begin{proof}
    For all $s\in G$ and all $x \in X$, $e_{x}$ is an eigenvector of $\rho(s)$ if and only if $\rho(s)e_{x} = e_{x}$, which is equivalent to $s*x = x$. Thus, $\chi_{X}(s) = tr(\rho_s)$ is the sum of the eigenvalues of $\rho_{s}$ counted with geometric multiplicity (the total number of eigenvectors among $(e_{x})_{x \in X}$), so $\chi_{X}(s) = |G(s)|$ as desired.  
\end{proof}

\section*{Problem 2.3}
\begin{proof}
(Uniqueness) Suppose $\rho', \rho^\#: G \to GL(V')$ are two linear representations such that $\langle \rho_s x, \rho'_s x'\rangle = \langle x, \rho_s^{t}\rho^\#_s x'\rangle = \langle x, x' \rangle \ \text{for all } s \in G, x \in V, x' \in V'$.
Then for all $s \in G, x \in V, x' \in V'$, $\langle x, \rho_s^{t}\rho^{\#}_s x'\rangle = \langle x, \rho_s^{t}\rho'_s x'\rangle = \langle x, x' \rangle \ \text{for all } s \in G, x \in V, x' \in V'$, so $\rho_s^{t}\rho'_s = \rho_s^{t}\rho^\#_s = Id_{V'}$. As a result $\rho^\# = \rho'_s = (\rho_s^{t})^{-1} = (\rho_s^{-1})^{t} = (\rho_{s^{-1}})^{t}$ for all $s \in G$, and $\rho' = \rho^\#$ as desired.

(Existence) Define $\rho'$ by $\rho'(s) := \rho(s^{-1})^t$ for all $s \in G$.
\end{proof}


\section*{Problem 2.4}
\begin{proof}
    Let $r, s \in G$. Then for all $f \in W$, $\rho_s\circ\rho_r f = \rho_{2, s} \circ \rho_{2, r} \circ f \circ \rho_{1, r}^{-1} \circ \rho_{1, s}^{-1} = \rho_{2, sr} \circ f \circ \rho_{1, sr}^{-1}$, so $\rho$ defines a linear representation.

    Since $\rho$ is isomorphic to $\rho_1' \otimes \rho_2$, $\chi_\rho = \chi_{1}^* \cdot \chi_2$.
\end{proof}
\end{document}